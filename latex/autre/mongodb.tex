\documentclass{article}
\usepackage[utf8]{inputenc}
\usepackage{graphicx}
\usepackage{subfig}
\usepackage{titlepic}

\usepackage{hyperref}
\hypersetup{
    colorlinks=true,
    linkcolor=red,
    filecolor=magenta,      
    urlcolor=cyan,
}

\title{mangoDB intro}
\author{tom.rsr }
\date{December 2018}
\titlepic{\includegraphics[width=\textwidth]{image/mongodb.jpg}}

\begin{document}
\begin{titlepage}
    \maketitle
    \section{Introduction}
    dans ce document nous allons parler de comment installer MongoDb, comment créer une base de données, comment l'interroger et modifier les données. Et je donnerais mon opinion sur ce système
\end{titlepage}
\newpage
\tableofcontents
\newpage

\section{Installation}
\subsection{télécharger la dernière version de MongoDB (v4.0.4)}
Pour télécherger MongoDB rendez vous à l'adresse :\\ \url{https://www.mongodb.com/download-center/community}
Choisissez la dernière version stable (current release)\\ choisissez le type d'installation que vous voulez sous Package (\hyperref[sec:MSI]{MSI} ou \hyperref[sec:ZIP]{ZIP}) et l'opérating système sur lequel vous êtes

\subsection{installer avec MSI}
\label{sec:MSI}
quand le téléchargement sera fini ouvrez le fichier .MSI 
\begin{enumerate}
\item appuyer sur Next
\item Accepter les termes et appuyer sur Next
\item Cliquez sur Complete
\item cliquez sur la case Installer MonDB as a Service pour la désactiver, puis sur Next
\item Cliquez sur Complete puis sur Next
\item Cliquez sur Complete puis sur Next
\item Cliquez sur Complete puis sur Next
\end{enumerate}
\subsection{installer avec ZIP}
\label{sec:ZIP}
une fois le téléchargement terminé placer le dossier par exemple sur le bureau, aller dans le dossier puis dans bin et vous pourrez démarrer le serveur en cliquant sur "mongod.exe".
\section{Création de la base de données}
\section{exemples }
\end{document}